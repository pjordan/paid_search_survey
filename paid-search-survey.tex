% v2-acmsmall-sample.tex, dated March 6 2012
% This is a sample file for ACM small trim journals
%
% Compilation using 'acmsmall.cls' - version 1.3 (March 2012), Aptara Inc.
% (c) 2010 Association for Computing Machinery (ACM)
%
% Questions/Suggestions/Feedback should be addressed to => "acmtexsupport@aptaracorp.com".
% Users can also go through the FAQs available on the journal's submission webpage.
%
% Steps to compile: latex, bibtex, latex latex
%
% For tracking purposes => this is v1.3 - March 2012

\documentclass[prodmode,acmtist]{acmsmall} % Aptara syntax

% Package to generate and customize Algorithm as per ACM style
\usepackage[ruled]{algorithm2e}
\renewcommand{\algorithmcfname}{ALGORITHM}
\SetAlFnt{\small}
\SetAlCapFnt{\small}
\SetAlCapNameFnt{\small}
\SetAlCapHSkip{0pt}
\IncMargin{-\parindent}

% Metadata Information
\acmVolume{9}
\acmNumber{4}
\acmArticle{39}
\acmYear{2013}
\acmMonth{3}

% Document starts
\begin{document}

% Page heads
\markboth{P.~R.~Jordan}{A Practitioner's Guide to Paid Search Literature}

% Title portion
\title{A Practitioner's Guide to Paid Search Literature}
\author{PATRICK R. JORDAN
\affil{Microsoft}}

\begin{abstract}
TODO: Abstract
\end{abstract}

\category{K.4.4}{Computers and Society}{Electronic Commerce}

\terms{Algorithms, Economics}

\keywords{Online advertising, paid search}

\acmformat{Patrick R. Jordan, 2013. A Practitioner's Guid to Paid Search Literature.}

\maketitle

\section{Introduction}

TODO

\section{History} % (fold)
\label{sec:history}

The primary origins of paid search are well documented \cite{Jansen:2008uq}.
OpenText Corporation, an early provider of search technology for Yahoo! in the mid-1990s, developed, arguably, the first paid search system.\footnote{see \url{http://www.opentext.com/2/global/company/company-history-95-98.htm}}
In contrast to banner advertising, where advertisers purchased slots within web pages where their ads were shown, OpenText allowed advertisers to position ads within search results.
Advertisers could select which user queries would trigger premium placement of their ads.
However, rather than place paid links around organic (non-paid) links, as current systems do, OpenText maintained a single set of links whose order could be manipulated as to favor paying advertisers.\footnote{see \url{http://news.cnet.com/2100-1023-215491.html}}
This early experiment in manipulating search results failed due to customer and industry complaint, and the project was scrapped. 
Following OpenText's failed venture, Goto.com pioneered a new system in Februrary 1998, whereby paid links appeared alongside, rather than within, the organic search results.\footnote{see \url{http://news.cnet.com/Pay-for-placement-gets-another-shot/2100-1023_3-208309.html}}

In a further departure from the existing online advertising systems, advertisers using paid search to place ads did not pay each time a user viewed an ad (pay per impression), rather each time a user clicked their link (pay per click).
In GoTo's system, each time a user queries the search engine, paid ad placement was determined by an auction run in real-time.
Advertisers submitted bids signifying their willingness-to-pay per click and selected which queries they would bid on by specifying keywords that the query must match.
Advertisers were ranked according to their bid and premium slots were allocated in accordance with that ranking. 

Goto.com was renamed Overture in 2001.
In February 2002, Overture changes the pricing rules of the auction: each winning advertiser pays the maximum of the next highest bid or the reserve price, termed \emph{generalized second price} (GSP).
In March 2002, Google modifies its existing AdWords system to conform to GSP, however their new system incorporates quality signals, such as the rate users click the ad (CTR), in addition to bids when determining ranking. 
Overture immediately files a patent infringement lawsuit against Google.


Subsequently, circa 2005, a few other entrants including Ask.com, LookSmart, and MSN/Live.com launch their own paid search services.
In 2007, Yahoo! Search Marketing, after aquiring Overture in 2003, switched from \emph{bid ordering} to \emph{revenue-ordering} (the product of CTR and bid) when its Panama system launched. 
In 2010, Microsoft and Yahoo! announced their search partnership, which combined two separate paid search marketplaces into one powered by Microsoft's Bing for algorithmic search and AdCenter (now Bing Ads) for paid search. 
GSP remains the dominant pricing rule in the current paid search marketplaces.

% section history (end)

\section{Marketplace Design} % (fold)
\label{sec:auction_and_marketplace_design}

In practice, paid search marketplaces have a number of strategic participants that include advertisers, publishers, agencies, and optimization platforms.  
Existing research into these markets typically abstracts the intermediate relationships and focuses principally on the interaction between endpoints: advertisers and publishers.  
Users, whose behavior is not considered strategic, play an important role in defining the value of the transactions in the market. 
Abstractly, advertisers derive value by inducing users to transact with them.  
Publishers provide an avenue for advertisers to influence users through ads on a search results page (SERP).  
In response, publishers charge for that influence.  Publishers determine how ads are allocated to available slots on the SERP and how much advertisers are charged for being allocated. 

In all major search engines, advertisers pay when a user clicks on their respective ad.  
When a click occurs, the user is directed to the advertiser’s landing page.  
Advertisers derive value from the subsequent interaction by the user---for instance, the user may purchase one of the advertiser’s products.  
Because individual users can behave differently, advertisers may ascribe a value to click that depends on the features of a particular user. 

One of the most important features of a particular user, in terms of estimating the value to an advertiser, is the query the user submitted to the search engine.  
In some cases, the query gives a strong signal of user intent.  
Because the value of a click may differ vastly according to a particular user and advertiser, publishers use auctions to determine allocation and pricing in the paid search marketplace. 

Abstractly, advertisers place bids in terms of their willingness to pay per click and the publisher determines which ads get allocated to which slots and at what contingent price.  
Because the user query can be so discriminating in terms of advertiser value, advertisers condition their decision to bid on a particular user on the keywords within the query.  
In this way, keywords define submarkets. 

In order to participate in the marketplace, advertisers define campaigns that consist of ads to be displayed for different keywords as well as their willingness to pay per click for the displaying the.  
In addition advertisers may specific budgets, which the publishers must respect, as well as other signals of value, such as conversion data. 

At this level, paid search can be seen as a stochastic game of severely incomplete and imperfect information.  
In order to make traction in understanding the strategic choices of participants, early research abstracted the scenario down to a single auction with complete information. 

\subsection{Initial Models and Generalized Second Price} % (fold)
\label{sub:initial_models_and_generalized_second_price}

In the initial models of paid search (dependently introduced by Varian and EOS), there are a set of advertisers, [Equation], and a set of available slots, [Equation].  
The user’s click behavior is defined probabilistically: a user clicks on advertiser [Equation]’s ad in position [Equation] with probability [Equation].  
In these models, the click probability (or click-through-rate) [Equation] is assumed to be separable, in that  [Equation] where [Equation] is the position effect for slot [Equation] and [Equation] is the advertiser effect for advertiser [Equation].  
Slots are ordered in terms of [Equation] with [Equation].    
Finally, each advertiser [Equation] submits a bid [Equation] to the publisher that represents [Equation]’s willingness-to-pay for a click. 

[Insert figure illustrating the model.] 

The mechanics of the GSP auction are as follows.  
Each advertiser’s ad is scored according to [Equation].[Footnote]  
Advertisers are ranked according to [Equation], with [Equation] denoting the advertiser at index [Equation] in the ranking.  
Slots are allocated according to the rank with slot [Equation] going to advertiser [Equation].  
If the ad is clicked, advertiser [Equation] is charged a price of 

[Equation] 

%TODO for Thursday: 
%Describe the characterization of equilibria 
%Connect GSP equilibria with other notions from Varian and EOS 
%Lead into structural and other theoretical variants model

% subsection initial_models_and_generalized_second_price (end)

\subsection{Theoretical Variants} % (fold)
\label{sub:theoretical_variants}

TODO

% subsection theoretical_variants (end)

\subsection{Structural Model} % (fold)
\label{sub:structural_model}

TODO

% subsection structural_model (end)
% section auction_and_marketplace_design (end)

\section{Query Understanding} % (fold)
\label{sec:query_understanding}

TODO

% section query_understanding (end)

\section{Ad Selection and Relevance Modeling} % (fold)
\label{sec:ad_selection_and_relevance_modeling}

TODO

% section ad_selection_and_relevance_modeling (end)

\section{User Response Modeling} % (fold)
\label{sec:user_response_modeling}

TODO

% section user_response_modeling (end)

\section{Advanced Topics} % (fold)
\label{sec:advanced_topics}

TODO

% section advanced_topics (end)

\subsection{Exploration and Exploitation} % (fold)
\label{sub:exploration_and_exploitation}
TODO

% subsection exploration_and_exploitation (end)

\subsection{Reserve Pricing} % (fold)
\label{sub:reserve_pricing}

TODO

% subsection reserve_pricing (end)

\subsection{Budget Optimization} % (fold)
\label{sub:budget_optimization}

TODO

% subsection budget_optimization (end)

\subsection{Modeling Second-Order Effects} % (fold)
\label{sub:modeling_second_order_effects}

TODO

% subsection modeling_second_order_effects (end)

\subsection{Whole-Page Optimization} % (fold)
\label{sub:whole_page_optimization}

TODO

% subsection whole_page_optimization (end)

\subsection{Externalities} % (fold)
\label{sub:externalities}

TODO

% subsection externalities (end)

\section{Conclusion} % (fold)
\label{sec:conclusion}

TODO
% section conclusion (end)

% Acknowledgments
\begin{acks}
The author would like to thank Dr. J.C. Mao, Microsoft.
\end{acks}

% Bibliography
\bibliographystyle{ACM-Reference-Format-Journals}
\bibliography{paid-search-survey}
                             % Sample .bib file with references that match those in
                             % the 'Specifications Document (V1.5)' as well containing
                             % 'legacy' bibs and bibs with 'alternate codings'.
                             % Gerry Murray - March 2012

% History dates
\received{February 2007}{March 2009}{June 2009}

\end{document}
% End of v2-acmsmall-sample.tex (March 2012) - Gerry Murray, ACM


